%%%%%%%%%%%%%%%%%%%%%%%%%%%%%%%%%%%%%%%%%
% Twenty Seconds Resume/CV
% LaTeX Template
% Version 1.1 (8/1/17)
%
% This template has been downloaded from:
% http://www.LaTeXTemplates.com
%
% Original author:
% Carmine Spagnuolo (cspagnuolo@unisa.it) with major modifications by 
% Vel (vel@LaTeXTemplates.com)
%
% License:
% The MIT License (see included LICENSE file)
%
%%%%%%%%%%%%%%%%%%%%%%%%%%%%%%%%%%%%%%%%%

%----------------------------------------------------------------------------------------
%	PACKAGES AND OTHER DOCUMENT CONFIGURATIONS
%----------------------------------------------------------------------------------------

\documentclass[letterpaper]{twentysecondcv} % a4paper for A4

%----------------------------------------------------------------------------------------
%	 PERSONAL INFORMATION
%----------------------------------------------------------------------------------------

% If you don't need one or more of the below, just remove the content leaving the command, e.g. \cvnumberphone{}

\profilepic{sourabh} % Profile picture

\cvname{Sourabh} % Your name
\cvjobtitle{Aggarwal}
\cvdate{}
\cvaddress{}
\cvnumberphone{+91 7986311572} % Phone number
\cvsite{https://github.com/sourabh2311} % Personal website
\cvmail{111601025@smail.iitpkd.ac.in} % Email address

%----------------------------------------------------------------------------------------

\begin{document}

%----------------------------------------------------------------------------------------
%	Education 
%----------------------------------------------------------------------------------------

\education{
    B. Tech. CSE IIT Palakkad | 2020 | CGPA: 9.69/10 (Current)

    Class XII Sri Guru Ram Rai Public School | 2015 | 81\%

    Class X Delhi Public School Jhammat Ayali Kalan Ludhiana | 2013 | 78\%
} % To have no education section, just remove all the text and leave \education{}

%----------------------------------------------------------------------------------------
%	 Extra Curricular
%----------------------------------------------------------------------------------------

% Skill bar section, each skill must have a value between 0 an 6 (float)
\skills{}

%------------------------------------------------

% Skill text section, each skill must have a value between 0 an 6
\skillstext{}

%----------------------------------------------------------------------------------------
\ExtraCurricular{
    Actively setting contest problems in Institute's Coding Club.
 \newline \newline Have won medals in various BasketBall Competitions like Takshilla.
 \newline \newline Participated in various Institute's Cultural events like "Ek Bharat Shrestha Bharat".
}

\makeprofile % Print the sidebar

%----------------------------------------------------------------------------------------
%	 INTERESTS
%----------------------------------------------------------------------------------------

\section{Technical Skills}

\begin{twentyshort} % Environment for a short list with no descriptions
    \twentyitemshort{\href{https://github.com/sourabh2311/Competitive-Programming}{Competitive Programming}}{ Have solved many problems and have participated \newline in various algorithmic competitions}
    \twentyitemshort{Programming Languages}{C, C++, C\#, Dart, MIPS, Prolog, Python, Rust, \newline   \ Standard ML, Verilog, Python GUI Library Tkinter, \newline Latex, Markdown}
    \twentyitemshort{ML And AI}{Have done various online courses and projects in \newline Artificial Intelligence \& Machine Learning.}
    \twentyitemshort{Game Development}{Cross Platform Game Development with Unity Game Engine}
    \twentyitemshort{Photo Editing}{Photo Editing with Photoshop}

\end{twentyshort}


\section{Positions of Responsibility}

\begin{twentyshort} % Environment for a short list with no descriptions
    \twentyitemshort{1}{Club Head of Institute's Coding Club}

    \twentyitemshort{2}{Class Representative for 2 Semesters}
\end{twentyshort}

\section{Achievements}

\begin{twentyshort} % Environment for a short list with no descriptions
    \twentyitemshort{Chine Youth Delegation}{Selected by the Ministry of Youth Affairs and Sports, Govt. of \newline India among 200 students to
    represent India as a \newline youth delegate in the "Indian Youth Delegation \newline to China - 2018" (3rd
    July 2018 - 10th July 2018).}

    \twentyitemshort{Academic}{Received Certificate of Academic Excellence for \newline highest CGPA in my branch}
    \twentyitemshort{Competitive Programming}{ Secured All India Rank 20 in Preliminary 
      ICPC 2018 \newline \href{https://www.hackerearth.com/challenges/competitive/JNJ-3addresscode-2019/leaderboard/}{Secured All India Rank 86 in a contest conducted by Johnson \newline \& Johnson}}
\end{twentyshort}


\section{Projects}

\begin{twenty} % Environment for a list with descriptions
    \twentyitem{Sem VI}{\href{https://github.com/nikhilyadv/DBMS-Lab-Project}{Database Project}}{Database}{My team created a database with GUI written \newline using python library Tkinter and Database written \newline in SQL (MariaDB) for amazon like marketplace.}
    \twentyitem{Sem VI}{\href{https://bitbucket.org/sourabh2311/111601025-compiler-lab/src/master/Project/}{Compiler}}{Compiler}{Wrote a compiler to compile Tiger language to \newline MIPS using Standard ML, ML-Lex, ML-YACC. \newline Since repository is private due to academic \newline reasons, email me for an access.}
    \twentyitem{Sem V}{\href{https://docs.google.com/document/d/1UtRv_9zUFreQW2REPL6swftzBgMALkfyG2-4s9Al1J8/edit?usp=sharing}{Predicting Taxi Travel Time}}{ML}{Analysed various approaches to predict a trip's travel time with their pros and cons, given
    the partial taxi trajectory of various taxis running in the city of Porto.}
    \twentyitem{Sem IV}{\href{https://github.com/satvik007/Tic-tac-toe}{Tic Tac Toe Game With AI on FPGA}}{AI, Verilog, FPGA}{Implemented a Tic Tac Toe game \newline between a human player and an AI. The main \newline code is
    written in Verilog language. The code \newline for the AI part was generated with C++ using \newline concepts
    of graph theory and dynamic programming.}
    \twentyitem{Sem IV}{\href{https://docs.google.com/document/d/1sE1sI62jE6UjglZAsnLe_8nbFOwQhV9QEffgOyZr4vg/edit?usp=sharing}{Surveillance Using Motion Detection}}{Signals And Systems}{Came up with a system capable of recognizing motion and storing the signals after
    classifying the digital signal as interesting or uninteresting. It takes into consideration
    practical scenarios such as small noise and ambient lighting.}
    \twentyitem{}{}{}{}

\end{twenty}

\end{document} 
